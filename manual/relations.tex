\section{Domains, symbols, and relations}\label{sec:relations}

It is possible to equip domains with extra structure, such as an order. LOIS includes
an internal solver for homogeneous with extension bounds \cite{lois-sat}; also,
several external solvers can be used for the structure $(\bbN, +)$ or $(\bbR, +, *)$.
Such an extra
structure does not have to be immediately declared when the domain is created; instead,
at any time the programmer can create an object from one of the subclasses of the class
{\tt Relation}, for example with the following declaration:

\begin{lstlisting}
RelOrder newOrder(" GT ", " LEQ ", " MAX ", " MIN ");
\end{lstlisting}

Terms representing elements of any domain can now be compared with respect to
{\tt newOrder} (a dense total order without endpoints) using the methods {\tt rbool less(const term\& a, const term\& b)} or
{\tt rbool leq(const term\& a, const term\& b)}. We can also find the maximum and minimum
of two elements, using the methods {\tt term max (const term\& a, const term\& b)}
and {\tt term min (const term\& a, const term\& b)}.

When the formulas are displayed on the screen, one of the symbols given in the
declaration of {\tt newOrder} is used. Symbols are given as objects of class
{\tt symbol}; a conversion from {\tt const char*} and {\tt std::string} to {\tt symbol}
is provided, but the class {\tt symbol} also allows to use symbols which are displayed
differently depending on the context. For example, the object {\tt sym} contains 
many symbols which are either used by LOIS itself or considered useful in applications,
and methods {\tt useUnicode()}, {\tt useLaTeX()},
and {\tt useASCII()}, which sets all the symbols to use the given format. The following
symbols are defined in the object {\tt sym}:

\begin{itemize}
\item Logical symbols:
{\tt exists}, {\tt forall}, {\tt \_and}, {\tt \_or}, {\tt eq}, {\tt neq}, {\tt in}

\item Basic set symbols:
{\tt emptyset}, {\tt ssunion} (an union of set-builder expessions), 
{\tt leftbrace}, {\tt rightbrace}, {\tt sfepipe} (a separator between the value
and the context in the set-builder expression), {\tt sfecomma} (a separator between
in the context of a set-builder expression), {\tt pseudo} (operator which extracts
a single element from a set, for the purpose of displaying {\tt lelem}s)

\item Relational and functional symbols:
{\tt leq}, {\tt geq}, {\tt greater}, {\tt less}, {\tt max}, {\tt min},
{\tt plus}, {\tt times}, {\tt minus}, {\tt divide}, {\tt edge}, {\tt noedge},
{\tt arrow}, {\tt noarrow}
\end{itemize}

LOIS declares the \emph{main order}, which is defined with the following line:

\begin{lstlisting}
  mainOrder = new RelOrder(sym.greater, sym.leq, sym.max, sym.min);
\end{lstlisting}

The C++ opperators {\tt <}, {\tt >}, {\tt <=} and {\tt >=}, when used on terms,
are defined as referring
to {\tt mainOrder}. When two orders (say, {\tt mainOrder} and {\tt newOrder}) are
defined and used in the same formula and over the same domain, they are 
considered to be unrelated -- intuitively, two random orders have been independently
chosen over our domain. In general, the same rule of independence is used when
defining multiple relations; in many cases, this yields oligomorphic homogeneous structures,
which still have a decidable first order theory \cite{lois-sat}. The algorithm used by
LOIS for deciding the satisfiability only takes into account relations which are actually
used in the given formula; so, the existence of {\tt mainOrder}, even if it is not used,
won't make the computations slower.

The following subclasses of {\tt Relation} are available:

\begin{itemize}
\item {\tt RelOrder}, explained above. 
The constructor of {\tt RelOrder} has
four symbol arguments: {\tt greater}, {\tt leq}, {\tt max}, and {\tt min}.
{\tt RelOrder} has four methods: {\tt less}, {\tt leq}, {\tt min}, and {\tt max}.

\item {\tt RelBinary}, which creates a random binary relation over the domain.
The constructor of {\tt RelOrder} has two symbol arguments {\tt inrel} (in the relation)
and {\tt notinrel} (not in the relation), as well as two parameter {\tt l} and {\tt s},
which define the properties of relation to be created. The parameter {\tt l} can take
the value of {\tt lmNoLoops} (irreflexive), {\tt lmAllLoops} (reflexive), or 
{\tt lmPossibleLoops} ($a$ can be in the relation with itself or not). The parameter
{\tt s} can take the value of {\tt smSymmetric}, {\tt smAsymmetric}, or {\tt smAntisymmetric}.
In all cases, with probability 1 we get the same graph
(up to isomorphism), which has a decidable first order theory due to being
homogeneous and oligomorphic (see \cite{lois-sat} for details); when choices
which leave more possibilities ({\tt lmPossibleLoops}, {\tt smAsymmetric})
are chosen, the internal solver has to consider all of them to verify the satisfiability
of formulas.
The relation is accessed with {\tt rbool operator () (const term\& a, const term\& b)}.

\item {\tt RelUnary}, which creates a random partition of the domain. The constructor is
{\tt RelUnary(symbol rel, int n)}, where every element of the domain is randomly
assigned to one of the {\tt n} parts. Operator {\tt rbool operator () (const term\& a, int v)}
is used to check whether {\tt a} is in the part number {\tt v} (0-based), and
for convenience, the method
{\tt rbool together(const term\& a, const term\& b)} checks whether {\tt a} and {\tt b}
are in the same parts.

\item {\tt RelTree}, which defines a homogeneous tree structure.
This is an infinite tree, where every two elements have the least common ancestor,
and it is dense without endpoints, that is, if $u$ is an ancestor of $v$, then
there is a $w$ such that $w$ is ancestor of $v$ and $u$ is ancestor of $w$, and
there is an ancestor of $u$ and a descendant of $w$; furthermore, there is infinite
branching, that is, if $v_1, \ldots, v_k$
are descendants of $u$, then there is $v$ such that the least common ancestor
of $v$ and $v_i$ for $i=1, \ldots, k$ is $u$. This structure showcases the fact
that, in a homogeneous structure with extension bound, the isomorphism type of
$\{a_1, \ldots, a_n\}$ might be defined not only by relations
on these elements -- we also need to check the relations of terms (in this case,
using lca).
% \cite{lois-sat}.
Methods {\tt anceq} (``ancestor or equal'') and {\tt lca} are used to
access the relations and functional symbols. The constructor has three symbol arguments
{\tt opanceq}, {\tt opnotanceq}, and {\tt oplca}, which correspond to the methods.

\item {\tt RelInt} and {\tt RelReal}, which define the set of integers and reals,
respectively. The constructor defines
the symbols related to the order, and also {\tt opplus}, {\tt optimes}, {\tt opminus},
and {\tt opdivide}. The term {\tt constant(Domain *d, int i)} is used for integer
constants, and there are also methods {\tt plus}, {\tt times}, {\tt minus}, and {\tt
divide} for the basic operations.
Note that these are not $\omega$-categorical,
and thus they are not compatible with other relations, and the internal solver does
not work with them -- an external solver is required \cite{lois-sat}.
Use the namespace {\tt orderedfield\_ops} from {\tt loisextra.h} to conveniently 
bind the C++ operators (\texttt{+ - * /}) to methods of the given {\tt RelInt} or
{\tt RelReal}.
\end{itemize}

Note that the domain for a given {\tt term} can be obtained with the method {\tt getDom()}.
