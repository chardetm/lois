\section{Syntax and features}\label{app_pseudo}
In this appendix, we list the key constructions available in LOIS. 
% Some of these constructions have
% already appeared in Section~\ref{sec_tut}, but
% we repeat them here for completeness and a more
% structured presentation.
Also, we 
give an informal description of some constructs, 
which is sufficient for understanding basic programs in LOIS
on an intuitive level. This should not be confused with the formal semantics, which is given in 
\cite{lois-sem}.

\subsection{Types}
LOIS allows the programmer to manipulate sets, which are represented by the type {\tt lset}.


%The distinction between lvalues and rvalues is for technical reasons (see Section~\ref{sec_semantics}).

 There is a polymorphic type {\tt elem} representing elements of sets, which should be used as the type of a control variable in a {\tt for} loop. 
 The programmer can define set elements basing on any C++ type which has the following basic operations
defined: variable substitution (create a copy with the variable changed),
output (to a C++ stream),
equality test, and checking whether it depends on a given variable
(required for optimization). Integers, pairs (type {\tt elpair}), and tuples (type {\tt eltuple}) are based on the standard C++ types, and are defined as  {\tt int}, {\tt pair<elem, elem>} and {\tt vector<elem>}, respectively. 
Terms are represented by the type {\tt term}.

The type {\tt lelem} represents element of a set. 
While {\tt elem} contain elements of a fixed type (e.g., a pair), {\tt lelem} 
represents {\it piecewise elements}, whose type may depend on the variables
in the context -- intuitively,
their type may different in different pseudoparallel threads.
They are described in detail in Section \ref{app_piecewise} below. Also, {\tt lnum<T>}
is a wrapper for piecewise numbers -- they are described in
Subsection \ref{app_piecewise_num} below.

Finally, there are two types for representing booleans whose values may depend on
variables in their current (inner) context, {\tt lbool} and {\tt rbool}. 
Type {tt rbool} simply represents a formula, and {\tt lbool} has an inner context,
allowing it to work according to the pseudoparallel semantics.

\myparagraph{Conversions}
There is an assignment operator for assigning a variable of type {\tt relem} to a 
variable of type {\tt lelem}. Thanks to this and the C++ conversion mechanism, whenever
a variable of type {\tt lelem} is used in place where a variable of type {\tt relem} is
expected, an automatic conversion is performed by the compiler (and similarly for the
other pairs of lvalue and rvalue types).
To convert a {\tt relem e} to a {\tt elem}, \verb-for(elem x: newSet(e))- can be used.
Also, pairs, tuples, terms, integers, and sets are automatically converted to {\tt elem}s.
A variable {\tt x} of type {\tt elem} can be cast to a set, a pair, a tuple, a term, or an integer, using the operations {\tt asSet(x)}, {\tt as<elpair>(x)}, {\tt as<eltuple>(x)}, {\tt as<term>(x)}, and {\tt as<int>(x)}, respectively. It can be also tested whether {\tt x} is a pair, by the operation {\tt is<elpair>(x)} (and similarly for terms, tuples and integers) which returns a value of type {\tt bool}. 

\myparagraph{Sets with type checking}
The types \verb-lsetof<T>- is used instead of \verb-lset- 
to create sets whose all elements are of type \verb-T- (typically, 
\verb-term-, \verb-elpair-, \verb-eltuple-, \verb-lset-, \verb-int-, or \verb-lsetof<U>-). 
This is a wrapper around \verb-lset- which allows only adding elements
of type \verb-T-; moreover, the \verb-for- loop iterates over type \verb-T- instead
of the polymorphic \verb-elem-. 
% For example: \incc{setof}
This improves readability and provides type checking.


% \subsection{Syntax}

\subsection{Flow control}
LOIS has the following constructs constructs for flow control:
\begin{itemize}
\item The conditional {\tt If (cond) I}, where {\tt cond} is a condition of type {\tt rbool} and {\tt I} is an instruction (the body) which is to be executed if the condition is satisfied. There is also a variant 
{\tt Ife (cond) I else J}, where the instruction~{\tt J} is to be executed if the condition fails. {\tt If} and {\tt While} are macros; they should be used with LOIS conditions, rather than the normal {\pcif} and {\while}.
\item The looping construct {\tt While (cond) I}, where {\tt cond} and {\tt I} are as above.
\item The \emph{hybrid pseudoparallel} looping construct {\tt for (elem x:X) I},
where {\tt x} is the name of the introduced {control variable}, {\tt X} is the set (of type {\tt lset}) over which it ranges, and {\tt I} is an instruction (the body of the loop).
\item The fully pseudoparallel looping can be achieved in LOIS using the construct
{\tt for(relem e: fullypseudoparallel(X))}.
\end{itemize}

Furthermore, functions and recursion from C++ can be used. 
Note that since the {\tt for} loop in LOIS is
defined in a hybrid way \cite{lois-sem}, using the {\tt return}, {\tt break},
or {\tt continue} statement inside a loop will cause LOIS to stop processing the
set, and thus unintuitive behavior. The recommended approach is to create a value
(say, {\tt lelem ret}) to represent the returned value, set its value in the loop,
and then {\tt return ret}) after the loop ends.

\subsection{Operations} 
The basic operations on sets (\verb-lset-) are defined as follows. 
In the list, we denote sets (\verb-lset-) with \verb-X- and \verb-Y-, elements 
(\verb-lelem-) with \verb-x-, and set lvalues (\verb-lset-) as \verb-Z-.

Note that the assignment and compound operators (\verb-= += |= &= &=~-)
are defined according to the pseudo-parallel semantics \cite{lois-sem}. That means, for each
valuation $v$ of the inner context of $Z$, the operation is performed on $Z_v$
in parallel for each valuation $w$ of the current context which extends $v$.
Also note that adding elements to a set is much more efficient than removing
(which basically loops over the set and keeps only the elements which are not
to be removed).

\vskip 1em

\begin{tabular}{ll}
\verb-X==Y-&set equality \\
\verb-X!=Y-&set inequality \\
\verb-X&Y-&set intersection \\
\verb-X&~Y-&set difference \\
\verb-X|Y-&set union \\
\verb-X*Y-&Cartesian product (\verb-elpair- used for the pairs)\\
\verb-cartesian({X,Y,Z,...})-&Cartesian product (\verb-eltuple- used for the tuples)\\
\verb-subseteq(X,Y)-&\verb-X- is a subset of \verb-Y- \\
\verb-memberof(X,x)-&\verb-x- is an element of \verb-X- \\
\verb-newSet()-&create an empty set\\
\verb-newSet(x)-&create the set $\{\verb-x-\}$\\
\verb-newSet(x,y)-&create the set $\{\verb-x-, \verb-y-\}$\\
\verb-newSet({x,y,...})-&create a set with the given elements\\
\verb-extract(X)-&extract the single element of a set\\
\verb-Z=X-&set \verb-Z- to \verb-X-\\
\verb-Z+=x-&add \verb-x- to the set \verb-Z-\\
\verb|Z-=x|&remove \verb-x- from the set \verb-Z- \\
\verb-Z|=X-&add all elements of \verb-X- to the set \verb-Z-\\
\verb-Z&=X-&remove all elements of \verb-Z- which are not elements of \verb-X-\\
\verb-Z&=~X-&remove all elements of \verb-Z- which are elements of \verb-X- \\
\end{tabular}

\subsection{The underlying structure}
To create an infinite set, construct an object of the class {\tt Domain};
this object's method {\tt getSet()} returns the underlying set, of type {\tt lset}.
It is possible to create multiple domains.

All domains are automatically equipped with equality ({\tt ==}, {\tt !=}) and 
a dense total order, accessed with the usual operators {\tt <}, {\tt >}, {\tt <=}, {\tt >=}.

LOIS supports domains which have more structure, for example, with two independent dense
total orders. This is explained in detail later (Section \ref{sec:relations}).

\subsection{Quantifier macros}
LOIS defines macros \verb-FORALL-, \verb-EXISTS-, \verb-FILTER-, \verb-MAP-, 
and \verb-FILTERMAP-, allowing
the programmer to construct formulas with quantifiers, and sets, intuitively. 
\verb-FILTER-$(x, A, \phi(x))$ corresponds to the mathematical set
$\{x | x\in A, \phi(x)\}$, \verb-MAP-$(x, A, v(x))$ corresponds to
$\{v(x) | x \in A\}$, and \verb-FILTERMAP-$(x, A, \phi(x), v(x))$ corresponds to
$\{v(x) | x\in A, \phi(x)\}$. All these macros are defined 
 using the for loop,
and are therefore redundant in terms of expressive power.

\subsection{Displaying the current context, and naming variables}
The variable \verb-currentcontext- of type \verb-contextptr- contains a pointer
to the current context on the stack, while \verb-emptycontext- is the pointer
to an empty stack. (See \cite{lois-sem} for the discussion of contexts and stacks.)
Use \verb-cout << c- to display the difference between the
\verb-currentcontext- and \verb-c-. Thus, the following will display $\inco{displaycontext}$:

\incc{displaycontext}

Note how we have named the variables occuring in terms \verb-a- and \verb-b- 
to $a$ and $b$ -- otherwise LOIS would not know how the programmer named them,
so it would generate its own names, which would be probably some random letters
instead of $a$ and $b$.

Other functions working with contexts include {\tt branchset(contextptr anccontext)}
(which returns the set of all pseudoparallel threads since {\tt anccontext}), and
{\tt getorbit} (see {\tt loisextra.h}), which returns the orbit of a given element
(see \cite{lois-sat}), treating the variables in the given ancestor context as fixed.

\subsection{Declaring atoms and axioms}

One can declare atoms and axioms, like in the following:
\incc{declareatom}

Create an object of type \verb-declareatom- to name one element of the domain.
Create an object of type \verb-axiom- to add an axiom, for example, that the
declared atoms are not equal (this is not given). This is implemented by
pushing the respective variables and constraints on the stack. A careful reader
will notice that the real effect is exactly the same as would be obtained by using
\verb-for(a1,A) for(a2,A) If(a1!=a2)-. However, when the programmer wants simply to
select some elements of $\bbA$, these constructs are more intuitive than loops and
conditionals.


% Thus, for example, the following outputs $\inco{quantifiers}$
% (which is not simplified in this case, because we asked to output the formula, not
% to evaluate it):
%
% \incc{quantifiers}
%
% And the following outputs $\inco{mapfilter}$:
%
% \incc{mapfilter}

\subsection{Choosing the solver}
The global variable {\tt solver} of type {\tt solveptr} describes the solver currently
used by LOIS. The following solvers, or solver combinations, are available:

\begin{itemize}
\item {\tt solverCrash()} which cannot solve anything (it just crashes).
\item {\tt solverBasic()} which can solve only the trivial cases.
\item {\tt solverExhaustive(int t, bool v)} calls the internal solver. The number 
{\tt t} corresponds to the number of tries (possible valuations) after which the
internal solver gives up, and {\tt v} is the verbosity.
\item {\tt solverSMT()}, {\tt solverCVC()} and {\tt solverSPASS()} call the given
external solver. They accept an optional {\tt std::string} argument, which is the path to the solver
executable.
\item {\tt solverIncremental(std::string)} uses Z3's incremental solving feature.
\item {\tt solverCompare(std::initializer\_list<solveptr> p)} compares the results of two or more solvers,
and checks for inconsistencies.
\item {\tt solverVerbose} and {\tt solverNamed(std::string n, solveptr s)} are wrappers 
around solvers which provide extra diagnostic information.
\item {\tt solverStack(solveptr s, solveptr fallback)} calls the solver {\tt s},
then calls {\tt fallback} if it failed. It can be used, for example, to solve simple cases with the
basic or exhaustive solver, then proceed to verbose exhaustive or external solver for
the harder cases.
\end{itemize}

Call {\tt useDefaultSolver(int i, int j)} to use the default solver (queries with
complexity below $i$ are solved in a non-verbose way, then they are solved in a verbose
way, and the internal solver gives up at $j$ tries; also $j$ is used as the limit for
the number of tries for the simplification algorithm).

\subsection{Other functions}
Some other functions include:
\begin{itemize}
\item {\tt lset optimize(const lset\& x)}, which optimizes the set by removing repetitions
(each element in the returned set will be in exactly one set-builder expression).
\item {\tt lset optimizeType(const lset\& x, const lset\& type)}, which is a more efficient version of
{\tt optimize} for the case when we know that {\tt x} is a subset of a (simple) set
{\tt type}.
\item {\tt getsingletonset(const lset\& X)}, which returns the set of singletons of elements of X,
represented as a single set-builder expression.

\end{itemize}
