\vspace{-2em}
\section{Introduction}\label{sec_intro}

LOIS (Looping Over Infinite Sets) is a C++ library which allows working on 
\emph{definable} infinite sets in a natural way. We can create an infinite
\emph{domain}, let's say $\bbA$, possibly with some relational and functional symbols, 
and then use the \emph{pseudo-parallel} semantics to iterate over it in a natural way. 
This gives us new sets, for example $\{(x,y): x \in \bbA, y \in \bbA, x \neq y\}$,
which can be iterated over in turn, or checked for emptiness. A LOIS program will
work in finite time as long as the first order theory of $\bbA$ is decidable.
LOIS is open source, released under MIT license.

Its homepage is at
{\tt http://www.mimuw.edu.pl/\textasciitilde erykk/lois/},
and its GitHub repository is at
{\tt https://www.github.com/eryxcc/lois}.


This document is a technical description of LOIS, and thus, many theoretical details
have been omitted. See the papers on the website above for details about:

\begin{itemize}
\item the theoretical foundations: definable sets, homogeneous structures, \cite{lois-sat}
\item how the solvers (built-in, CVC4, Z3, SPASS) are used to make the computation
possible, \cite{lois-sat}
\item the results of our tests of internal and external solvers, \cite{lois-sat}
\item applications, \cite{lois-sat}
\item the novel pseudoparallel semantics which LOIS is using to handle the infinite sets.
\cite{lois-sem}
\end{itemize}

Note that, in the current prototype, all the core functionality works, but in case of
some functions, variants which accept more complex types (say, {\tt lnumof<T>} or {\tt
lsetof<T>} instead of the basic {\tt elem} or {\tt lset}) may be still missing,
or some obvious type casting might be missing, making C++ unable to guess
what typecasts should be used.
Usually
such more complex variants should be easy to write.

LOIS has been tested on machines with the following configurations:
\begin{itemize}
\item gcc version 4.4.3, architecture i686, Ubuntu Linux
\item gcc version 4.6.3, architecture x86\_64, Ubuntu Linux
\item gcc version 4.7.3, architecture x86\_64, PLD Linux
\end{itemize}

